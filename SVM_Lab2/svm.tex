\documentclass[fleqn]{llncs}
\usepackage{geometry}
\geometry{b5paper, left=2.5cm, right=2.5cm, top=2cm, bottom=2cm} % Adjust the margins as needed

%load needed packages
\usepackage{graphicx}
\usepackage{array}
\usepackage[utf8]{inputenc}
\usepackage{csvsimple}
\usepackage{hyperref}
\usepackage{caption}
\usepackage{float}
\usepackage{amsmath}
\usepackage{listings}
\usepackage{pgfplots}
\usepackage{xcolor}
\usepackage{amsmath}

\pgfplotsset{compat=newest}
\DeclareMathOperator{\sgn}{sgn}

\definecolor{codegreen}{rgb}{0,0.6,0}
\definecolor{codegray}{rgb}{0.5,0.5,0.5}
\definecolor{codepurple}{rgb}{0.58,0,0.82}
\definecolor{backcolour}{rgb}{0.953, 0.953, 0.948}

\lstdefinestyle{mystyle}{
	backgroundcolor=\color{backcolour},   
	commentstyle=\color{codegreen},
	keywordstyle=\color{magenta},
	numberstyle=\tiny\color{codegray},
	stringstyle=\color{codepurple},
	basicstyle=\ttfamily\footnotesize,
	breakatwhitespace=false,         
	breaklines=true,                 
	captionpos=b,                    
	keepspaces=true,                 
	numbers=left,                    
	numbersep=5pt,                  
	showspaces=false,                
	showstringspaces=false,
	showtabs=false,                  
	tabsize=2
}

\lstset {style = mystyle}

\setlength{\parindent}{0em}

\graphicspath{ {./images/} }

\setcounter{secnumdepth}{2}

\setcounter{MaxMatrixCols}{20}
\AtBeginEnvironment{pmatrix}{\setlength{\arraycolsep}{6pt}}

\begin{document}

\title{Practica 3 -- SVM}

\author{Fynn Meffert, José Antonio Chaires Monroy, Luis David López Magaña}
\institute{Aprendizaje Computacional. Universidad de Málaga.}

\maketitle

\vspace{1cm} % Space down the title

\section{Tarea a)}

\subsection{Script R Usado para las Computaciones}
\lstinputlisting[language=R, label=listing:code-a]{ml_svm/SVM_Lab2/part_a.R}

\newpage

\subsection{Los Vectores Soporte}
\begin{align*}
	\overrightarrow{v_1} =
		\begin{pmatrix}
			0 \\
			0 \\
		\end{pmatrix},
	\qquad
	\overrightarrow{v_2} =
		\begin{pmatrix}
			4 \\
			4 \\
		\end{pmatrix}
\end{align*}

\subsection{Los Valores del Kernel}
\begin{align*}
	K_{AA} = 0,
	\qquad
	K_{AB} = K_{BA} = 0,
	\qquad
	K_{BB} = 32,
\end{align*}

\subsection{El Ancho del Canal}
\begin{align*}
	ancho \approx 0.7071
\end{align*}

\subsection{El Vector $\overrightarrow{w}$}
\begin{align*}
	\overrightarrow{w} =
		\begin{pmatrix}
			-2 \\
			-2 \\
		\end{pmatrix}
\end{align*}

\subsection{El Vector $\overrightarrow{b}$}
\begin{align*}
	\overrightarrow{b} = 0
\end{align*}

\subsection{La Ecuación del Hiperplano y de los Planos de Soporte Positivo y Negativo}
\begin{align*}
	\begin{pmatrix}
		-2 \\
		-2 \\
	\end{pmatrix} * \overrightarrow{x} = 0,
	\qquad
	\begin{pmatrix}
		-2 \\
		-2 \\
	\end{pmatrix} * \overrightarrow{x} = 1,
	\qquad
	\begin{pmatrix}
		-2 \\
		-2 \\
	\end{pmatrix} * \overrightarrow{x} = -1
\end{align*}

\subsection{Determinación de la Clase}
\begin{align*}
	clase \left(
	\begin{pmatrix}
		5 \\
		6 \\
	\end{pmatrix} \right) = -1,
	\qquad
	clase \left(
	\begin{pmatrix}
		1 \\
		-4 \\
	\end{pmatrix} \right) = 1
\end{align*}


\newpage


\section{Tarea b)}

\subsection{Script R Usado para las Computaciones}
\lstinputlisting[language=R, label=listing:code-b]{ml_svm/SVM_Lab2/part_b.R}

\newpage

\subsection{Los Vectores Soporte}
\begin{align*}
	\overrightarrow{v_1} =
	\begin{pmatrix}
		2 \\
		0 \\
	\end{pmatrix},
	\qquad
	\overrightarrow{v_2} =
	\begin{pmatrix}
		0 \\
		0 \\
	\end{pmatrix}
	\qquad
	\overrightarrow{v_3} =
	\begin{pmatrix}
		1 \\
		1 \\
	\end{pmatrix}
\end{align*}

\subsection{Los Valores del Kernel}
\begin{align*}
	K_{AA} = 4,
	\qquad
	K_{AB} = K_{BA} = 0,
	\qquad
	K_{BB} = 0,
	\quad
	K_{BC} = K_{CB} = 0,
	\quad
	K_{CC} = 2
\end{align*}

\subsection{El Ancho del Canal}
\begin{align*}
	ancho \approx 1.898
\end{align*}

\subsection{El Vector $\overrightarrow{w}$}
\begin{align*}
	\overrightarrow{w} =
	\begin{pmatrix}
		0.9995 \\
		-0.3335 \\
	\end{pmatrix}
\end{align*}

\subsection{El Vector $\overrightarrow{b}$}
\begin{align*}
	\overrightarrow{b} = 0.3333
\end{align*}

\subsection{La Ecuación del Hiperplano y de los Planos de Soporte Positivo y Negativo}
\begin{align*}
	\begin{pmatrix}
		0.9995 \\
		-0.3335 \\
	\end{pmatrix} * \overrightarrow{x} + 0.3333 = 0,
	\qquad
	\begin{pmatrix}
		0.9995 \\
		-0.3335 \\
	\end{pmatrix} * \overrightarrow{x} + 0.3333 = -1
\end{align*}
\begin{align*}
	\begin{pmatrix}
		0.9995 \\
		-0.3335 \\
	\end{pmatrix} * \overrightarrow{x} + 0.3333 = -1
\end{align*}

\subsection{Determinación de la Clase}
\begin{align*}
	clase \left(
	\begin{pmatrix}
		5 \\
		6 \\
	\end{pmatrix} \right) = 1,
	\qquad
	clase \left(
	\begin{pmatrix}
		1 \\
		-4 \\
	\end{pmatrix} \right) = 1
\end{align*}

\newpage

\section{Tarea c)}

TODO

\subsection{Script R Usado para las Computaciones}
\lstinputlisting[language=R, label=listing:code-c]{ml_svm/SVM_Lab2/part_c_e1071.R}

\newpage

\subsection{Los Vectores Soporte}
\begin{align*}
	\overrightarrow{v_1} =
	\begin{pmatrix}
		2 \\
		0 \\
	\end{pmatrix},
	\qquad
	\overrightarrow{v_2} =
	\begin{pmatrix}
		0 \\
		0 \\
	\end{pmatrix}
	\qquad
	\overrightarrow{v_3} =
	\begin{pmatrix}
		1 \\
		1 \\
	\end{pmatrix}
\end{align*}

\subsection{Los Valores del Kernel}
\begin{align*}
	K_{AA} = 4,
	\qquad
	K_{AB} = K_{BA} = 0,
	\qquad
	K_{BB} = 0,
	\quad
	K_{BC} = K_{CB} = 0,
	\quad
	K_{CC} = 2
\end{align*}

\subsection{El Ancho del Canal}
\begin{align*}
	ancho \approx 1.898
\end{align*}

\subsection{El Vector $\overrightarrow{w}$}
\begin{align*}
	\overrightarrow{w} =
	\begin{pmatrix}
		0.9995 \\
		-0.3335 \\
	\end{pmatrix}
\end{align*}

\subsection{El Vector $\overrightarrow{b}$}
\begin{align*}
	\overrightarrow{b} = 0.3333
\end{align*}

\subsection{La Ecuación del Hiperplano y de los Planos de Soporte Positivo y Negativo}
\begin{align*}
	\begin{pmatrix}
		0.9995 \\
		-0.3335 \\
	\end{pmatrix} * \overrightarrow{x} + 0.3333 = 0,
	\qquad
	\begin{pmatrix}
		0.9995 \\
		-0.3335 \\
	\end{pmatrix} * \overrightarrow{x} + 0.3333 = -1
\end{align*}
\begin{align*}
	\begin{pmatrix}
		0.9995 \\
		-0.3335 \\
	\end{pmatrix} * \overrightarrow{x} + 0.3333 = -1
\end{align*}

\subsection{Determinación de la Clase}
\begin{align*}
	clase \left(
	\begin{pmatrix}
		5 \\
		6 \\
	\end{pmatrix} \right) = 1,
	\qquad
	clase \left(
	\begin{pmatrix}
		1 \\
		-4 \\
	\end{pmatrix} \right) = 1
\end{align*}


\section{Tarea d)}

\subsection{Script R Usado para las Computaciones}
\lstinputlisting[language=R, label=listing:code-d]{ml_svm/SVM_Lab2/part_d.R}

\newpage

\subsection{Los Vectores Soporte}
\begin{align*}
	\overrightarrow{v_1} =
	\begin{pmatrix}
		2 \\
		2 \\
	\end{pmatrix},
	\qquad
	\overrightarrow{v_2} =
	\begin{pmatrix}
		1 \\
		1 \\
	\end{pmatrix}
\end{align*}

\subsection{Los Valores del Kernel}
\begin{align*}
	\begin{pmatrix}
		8 & 32 & 24 & 32 &  8 & 32 & 24 & 32 &  4 &  0 & -4 &  0 \\
		32 &136 & 96 &120 & 32 &136 & 96 &120 & 16 &  4 &-16 & -4 \\
		24 & 96 & 72 & 96 & 24 & 96 & 72 & 96 & 12 &  0 &-12 &  0 \\
		32 &120 & 96 &136 & 32 &120 & 96 &136 & 16 & -4 &-16 &  4 \\
		8 & 32 & 24 & 32 &  8 & 32 & 24 & 32 &  4 &  0 & -4 &  0 \\
		32 &136 & 96 &120 & 32 &136 & 96 &120 & 16 &  4 &-16 & -4 \\
		24 & 96 & 72 & 96 & 24 & 96 & 72 & 96 & 12 &  0 &-12 &  0 \\
		32 &120 & 96 &136 & 32 &120 & 96 &136 & 16 & -4 &-16 &  4 \\
		4 & 16 & 12 & 16 &  4 & 16 & 12 & 16 &  2 &  0 & -2 &  0 \\
		0 &  4 &  0 & -4 &  0 &  4 &  0 & -4 &  0 &  2 &  0 & -2 \\
		-4 &-16 &-12 &-16 & -4 &-16 &-12 &-16 & -2 &  0 &  2 &  0 \\
		0 & -4 &  0 &  4 &  0 & -4 &  0 &  4 &  0 & -2 &  0 &  2 \\
	\end{pmatrix}
\end{align*}

\subsection{El Ancho del Canal}
\begin{align*}
	ancho \approx 1.414
\end{align*}

\subsection{El Vector $\overrightarrow{w}$}
\begin{align*}
	\overrightarrow{w} =
	\begin{pmatrix}
		1 \\
		1 \\
	\end{pmatrix}
\end{align*}

\subsection{El Vector $\overrightarrow{b}$}
\begin{align*}
	\overrightarrow{b} = 3
\end{align*}

\subsection{La Ecuación del Hiperplano y de los Planos de Soporte Positivo y Negativo}
\begin{align*}
	\begin{pmatrix}
		1 \\
		1 \\
	\end{pmatrix} * \overrightarrow{x} + 3 = 0,
	\qquad
	\begin{pmatrix}
		1 \\
		1 \\
	\end{pmatrix} * \overrightarrow{x} + 3 = 1,
	\quad
	\begin{pmatrix}
		1 \\
		1 \\
	\end{pmatrix} * \overrightarrow{x} + 3 = -1,
\end{align*}

\subsection{Plot}
\begin{figure}[H]
	\centering
	\includegraphics[width=0.8\textwidth]{d}
\end{figure}

\newpage

\section{Tarea e)}

\subsection{Script R Usado para las Computaciones}
\lstinputlisting[language=R, label=listing:code-e]{ml_svm/SVM_Lab2/part_e.R}

\newpage

\subsection{Los Vectores Soporte}
\begin{align*}
	\overrightarrow{v_1} =
	\begin{pmatrix}
		5 \\
		3 \\
	\end{pmatrix},
	\qquad
	\overrightarrow{v_2} =
	\begin{pmatrix}
		1 \\
		0 \\
	\end{pmatrix}
\end{align*}

\subsection{Los Valores del Kernel}
\begin{align*}
	\begin{pmatrix}
		34 & 60 & 49 & 75 &  5 &  3 & -3 & -5 \\
		60 & 106 & 87 & 133 &  9 &  5 & -5 & -9 \\
		49 & 87 & 73 & 111 &  8 &  3 & -3 & -8 \\
		75 & 133 & 111 & 169 & 12 &  5 & -5 & -12 \\
		5 & 9 & 8 & 12 &  1 &  0 &  0 & -1 \\
		3 & 5 & 3 & 5 &  0 &  1 & -1 &  0 \\
		-3 & -5 & -3 & -5 &  0 & -1 &  1 &  0 \\
		-5 & -9 & -8 & -12 & -1 &  0 &  0 &  1 \\
	\end{pmatrix}
\end{align*}

\subsection{El Ancho del Canal}
\begin{align*}
	ancho \approx 5
\end{align*}

\subsection{El Vector $\overrightarrow{w}$}
\begin{align*}
	\overrightarrow{w} =
	\begin{pmatrix}
		0.32 \\
		0.24 \\
	\end{pmatrix}
\end{align*}

\subsection{El Vector $\overrightarrow{b}$}
\begin{align*}
	\overrightarrow{b} = 1.32
\end{align*}

\subsection{La Ecuación del Hiperplano y de los Planos de Soporte Positivo y Negativo}
\begin{align*}
	\begin{pmatrix}
		0.32 \\
		0.24 \\
	\end{pmatrix} * \overrightarrow{x} + 1.32 = 0,
	\qquad
	\begin{pmatrix}
		0.32 \\
		0.24 \\
	\end{pmatrix} * \overrightarrow{x} + 1.32 = 1,
\end{align*}
\begin{align*}
	\begin{pmatrix}
		0.32 \\
		0.24 \\
	\end{pmatrix} * \overrightarrow{x} + 1.32 = -1
\end{align*}

\subsection{Determinación de la Clase}
\begin{align*}
	clase \left(
	\begin{pmatrix}
		4 \\
		5 \\
	\end{pmatrix} \right) = 1
\end{align*}

\subsection{Plot}
\begin{figure}[H]
	\centering
	\includegraphics[width=0.8\textwidth]{e}
\end{figure}

\newpage

\section{Tarea f)}

\subsection{Script R Usado para las Computaciones}
\lstinputlisting[language=R, label=listing:code-f]{ml_svm/SVM_Lab2/part_f.R}

\newpage

\subsection{Los Vectores Soporte}
\begin{tabular}{l l l c}
	Sepal.Length & Sepal.Width & Petal.Width & Species \\
	\hline
	5.1 & 3.3 & 0.5 & setosa \\
	4.5 & 2.3 & 0.3 & setosa \\
	4.9 & 2.4 & 1.0 & versicolor \\
	5.9 & 3.2 & 1.8 & versicolor \\
	6.3 & 2.5 & 1.5 & versicolor \\
	6.7 & 3.0 & 1.7 & versicolor \\
	6.0 & 2.7 & 1.6 & versicolor \\
	5.1 & 2.5 & 1.1 & versicolor \\
	4.9 & 2.5 & 1.7 & virginica \\
	6.0 & 2.2 & 1.5 & virginica \\
	6.2 & 2.8 & 1.8 & virginica \\
	6.1 & 3.0 & 1.8 & virginica \\
	7.2 & 3.0 & 1.6 & virginica \\
	6.3 & 2.8 & 1.5 & virginica \\
	6.0 & 3.0 & 1.8 & virginica \\
\end{tabular}

\subsection{Los Valores del Kernel}
\begin{figure}[H]
	\centering
	\includegraphics[width=0.9\textwidth]{f_kernel}
\end{figure}

\subsection{El Ancho del Canal}
\begin{align*}
	ancho \approx 0.3397
\end{align*}

\subsection{El Vector $\overrightarrow{w}$}
\begin{align*}
	\overrightarrow{w} =
	\begin{pmatrix}
		-5.469 \\
		0.2363 \\
		1.962 \\
		0.9185
	\end{pmatrix}
\end{align*}
\begin{tabular}{l l l c}
	Sepal.Length & Sepal.Width & Petal.Length & Petal.Width \\
	\hline
	-5.4691320 & 0.2362545 & 1.9623159 & 0.9175388 \\
\end{tabular}

\subsection{El Vector $\overrightarrow{b}$}
\begin{align*}
	\overrightarrow{b} =
	\begin{pmatrix}
		-1.474 & -0.2958 & 10.58
	\end{pmatrix}
\end{align*}

\subsection{La Ecuación del Hiperplano y de los Planos de Soporte Positivo y Negativo}
\begin{align*}
	\begin{pmatrix}
		-5.4691 \\
		0.2363 \\
		1.9623 \\
		0.9175
		\end{pmatrix} * \overrightarrow{x} + \begin{pmatrix}
		-1.474 & -0.2958 & 10.58
		\end{pmatrix} = 0,
\end{align*}
\begin{align*}
	\begin{pmatrix}
		-5.4691 \\
		0.2363 \\
		1.9623 \\
		0.9175
	\end{pmatrix} * \overrightarrow{x} + \begin{pmatrix}
		-1.474 & -0.2958 & 10.58
	\end{pmatrix} = 1
\end{align*}
\begin{align*}
	\begin{pmatrix}
		-5.4691 \\
		0.2363 \\
		1.9623 \\
		0.9175
	\end{pmatrix} * \overrightarrow{x} + \begin{pmatrix}
		-1.474 & -0.2958 & 10.58
	\end{pmatrix} = -1
\end{align*}


\end{document}
